\begin{theorem}[red]{Définition 3.1 : Série alternée}
    Soit $(u_n)$ une suite réelle. On dit que $\sum u_n$ est \textbf{alternée} s'il existe une suite réelle $(a_n)$ positive telle que pour tout $n \in \ensN$, $u_n = (-1)^n a_n$.
\end{theorem}

\begin{theorem}[orange]{Théorème 3.2 : Critère spécial des séries alternées}
    Soit $(u_n)$ une suite réelle alternée. Si $(|u_n|)$ est décroissante et converge vers 0, alors

    \begin{itemize}
        \item $\sum u_n$ converge.
        \item $\forall n \in \ensN, |R_n| \leq |u_{n + 1}|$ \hspace{5px} avec $(R_n)$ la suite des restes associée à $\sum u_n$.
    \end{itemize}
\end{theorem}

\begin{theorem}[black]{Exemple du critère spécial de convergence}
    Soit $\alpha \in \ensR$. Alors, $\displaystyle \sum \frac{(-1)^n}{n^\alpha}$ converge ssi $\alpha > 0$.
\end{theorem}

\begin{theorem}[blue]{Définition 3.3 : Convergence absolue}
    On dit qu'une série numérique $\sum u_n$ \textbf{converge absolument} si la série $\sum | u_n |$ converge.
\end{theorem}

\begin{theorem}[orange]{Théorème 3.4 : Théorème de convergence absolue}
    Soit $\sum u_n$ une suite numérique convergeant absolument. Alors, $\sum u_n$ converge.
\end{theorem}